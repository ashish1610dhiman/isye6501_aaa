% Options for packages loaded elsewhere
\PassOptionsToPackage{unicode}{hyperref}
\PassOptionsToPackage{hyphens}{url}
%
\documentclass[
]{article}
\usepackage{amsmath,amssymb}
\usepackage{lmodern}
\usepackage{iftex}
\ifPDFTeX
  \usepackage[T1]{fontenc}
  \usepackage[utf8]{inputenc}
  \usepackage{textcomp} % provide euro and other symbols
\else % if luatex or xetex
  \usepackage{unicode-math}
  \defaultfontfeatures{Scale=MatchLowercase}
  \defaultfontfeatures[\rmfamily]{Ligatures=TeX,Scale=1}
\fi
% Use upquote if available, for straight quotes in verbatim environments
\IfFileExists{upquote.sty}{\usepackage{upquote}}{}
\IfFileExists{microtype.sty}{% use microtype if available
  \usepackage[]{microtype}
  \UseMicrotypeSet[protrusion]{basicmath} % disable protrusion for tt fonts
}{}
\makeatletter
\@ifundefined{KOMAClassName}{% if non-KOMA class
  \IfFileExists{parskip.sty}{%
    \usepackage{parskip}
  }{% else
    \setlength{\parindent}{0pt}
    \setlength{\parskip}{6pt plus 2pt minus 1pt}}
}{% if KOMA class
  \KOMAoptions{parskip=half}}
\makeatother
\usepackage{xcolor}
\usepackage[margin=1in]{geometry}
\usepackage{graphicx}
\makeatletter
\def\maxwidth{\ifdim\Gin@nat@width>\linewidth\linewidth\else\Gin@nat@width\fi}
\def\maxheight{\ifdim\Gin@nat@height>\textheight\textheight\else\Gin@nat@height\fi}
\makeatother
% Scale images if necessary, so that they will not overflow the page
% margins by default, and it is still possible to overwrite the defaults
% using explicit options in \includegraphics[width, height, ...]{}
\setkeys{Gin}{width=\maxwidth,height=\maxheight,keepaspectratio}
% Set default figure placement to htbp
\makeatletter
\def\fps@figure{htbp}
\makeatother
\setlength{\emergencystretch}{3em} % prevent overfull lines
\providecommand{\tightlist}{%
  \setlength{\itemsep}{0pt}\setlength{\parskip}{0pt}}
\setcounter{secnumdepth}{-\maxdimen} % remove section numbering
\ifLuaTeX
  \usepackage{selnolig}  % disable illegal ligatures
\fi
\IfFileExists{bookmark.sty}{\usepackage{bookmark}}{\usepackage{hyperref}}
\IfFileExists{xurl.sty}{\usepackage{xurl}}{} % add URL line breaks if available
\urlstyle{same} % disable monospaced font for URLs
\hypersetup{
  pdftitle={Submission HW1},
  hidelinks,
  pdfcreator={LaTeX via pandoc}}

\title{Submission HW1}
\author{}
\date{\vspace{-2.5em}}

\begin{document}
\maketitle

Submission for HW1 \textbar{} ISYE 6501 \textbar{} Fall 22\\
- Ashish Dhiman \textbar{}
\href{mailto:ashish.dhiman@gatech.edu}{\nolinkurl{ashish.dhiman@gatech.edu}}
- Abhinav Arun \textbar{}\\
- Anshit Verma \textbar{}

\hypertarget{question-2.1}{%
\subsection{Question 2.1}\label{question-2.1}}

\textbf{Describe a situation or problem from your job, everyday life,
current events, etc., for which a classification model would be
appropriate. List some (up to 5) predictors that you might use.}

\hypertarget{example-1-credit-risk-evaluation}{%
\paragraph{Example 1: Credit Risk
Evaluation}\label{example-1-credit-risk-evaluation}}

Classification models find use cases across a spectrum of Credit Risk
functions. A typical example being classifying a particular transaction
as risky or otherwise basis which it is either declined or authorized.
Here risk implies that the individual might not be able to make the
required payments in the future, and therefore transactions made are
riskier.

Objective Function (\(Y\)): To classify individual transactions as risky
(decline) or not Predictor Variables (\(X_i\)): Some of the key
predictor variables given below:

Past Delinquency: Delinquency implies that the individual was unable to
keep up on his monthly payments previously and missed on his obligated
payments. This is a key marker for credit risk, and past delinquent
behavior hints towards potential future risk.

Credit Utilization: Credit Utilization is defined as the ratio of
current balance to overall credit limit accorded to the individual.
Suppose a individual has a credit line of \$10,000 and he already has
utilized \$ 9k of it. This individual is generally prone to more risk as
compared to the individual who only has utilized say \$ 2k of his \$ 10k
line.

Current Debt to Income ratio: This is a measure of Income to Debt
Capacity of the individual. In other words, it stacks up the overall
debt obligations of the individual across mortgage, auto loan, credit
cards etc., against his total income. If a larger part of an
individual's income is directed towards his debt payment, than he/she
again might be more susceptible to miss payments in the future and hence
is riskier.

Amount of Transaction: This is the dollar amount of the transaction.
Intuitively a transaction of \$20k is riskier than \$5, since even if
the individual misses his payment, the hit taken by the firm is
restricted to only \$5.

\end{document}
